\documentclass{article}

\nonstopmode
\usepackage{graphicx,hyperref,amsmath,natbib,bm,url}
\usepackage{float}
\usepackage{microtype, todonotes}
\usepackage[a4paper, text = {16.5 cm, 25.2 cm}, centering]{geometry}
%\usepackage[a4paper, text = {16.5 cm, 2.2 cm}]{geometry}
%\usepackage[compact, small]{titlesec}
\setlength{\parskip}{1.2ex}
\setlength{\parindent}{0em}
\clubpenalty = 10000
\widowpenalty = 10000
\title{The long dust and ashes}
\author{Johan Blakkisrud}


\begin{document}

\maketitle

\subsection*{Treatment}

Thank you for that kind introduction. 

I am going to present dosimetric results in my presentation titled \emph{"Pre-dosing with lilotomab prior to therapy with 177Lu-lilotomab satetraxetan significantly increases the ratio of tumour to red marrow absorbed dose in non-Hodgkin lymphoma patients"}
177Lu-lilotomab satetraxetan (commercially called Betalutin�) is a novel antibody radionuclide conjugate targeting the CD37 antigen expressed on malignant B-cells. 
177Lu-lilotomab satetraxetan is currently investigated in the multi-center phase 1/2 LYMRIT 37-01 study headed by Oslo University Hospital. Patients with relapsed indolent non-Hodgkin lymphoma are included.


\subsection*{Arms}

%The patients received either 10, 15 or 20 MBq/kg body mass of 177Lu-lilotomab satetraxetan.
%In the phase 1 study, four different treatment arms were investigated, seen here as separate horizontal time lines (arm 1, 2 and so forth)
%I appologize for the busy slide, but if we follow each time line we see that the patients gets pre-%treated with rituximab some weeks before the treatment. 
%Then directly prior to treatment, the patients are pre-dosed with either 40 mg of lilotomab, no pre-dosing, another round with rituximab, or 100 mg/m2 BSA of lilotomab.
%The rationale of this pre-dosing is to block readily avaliable CD37 expressing cells, to allow more labelled antibodies to the tumour.
%Then at time zero the patiens are treated with 177Lu-lilotomab satetraxetan
%What follows is then a series of post-therapy imaging with SPECT/CT.

Patients received 10, 15 or 20 MBq/kg 177Lu-lilotomab satetraxetan in the phase 1 study. 
Four different arms were investigated, in which the patients received different pre-treatment and pre-dosing regimens. 
Each arm is illustrated here as horizontal lines, it is a bit busy but if you follow each line we see that each arms contains pre-treatment of rituximab.
Then predosing with either 40 mg of lilotomab, no pre-dosing, pre-dosing with another round of rituximab, and lastly 100 mg lilotomab per square meter body surface area of lilotomab.
At time zero, all patients got administered 177Lu-lilotomab satetraxetan.

Then followed a series of SPECT/CT-scans, from which we calculated the tumour and red marrow absorbed doses.

In previous investigations of arm 1 and 2, we found a significantly lower red marrow dose in arm 2 compared to arm 1.
The aim of our study was to investigate how the pre-dosing with lilotomab changed the uptake in the red marrow and tumors across all four arms.

\subsection*{SPECT/CT}

Here we see sagital images of one patient from each arm, four days post injection.
The patients received the same amount of activity per kilogram body mass
If we observe the spine, a reduction in red marrow uptake can be observed for patients in arm 1 and 4 compared to patients that did not receive pre-dosing with lilotomab (arm 2+3).
%The patients received the same amount of activity per kilogram body mass.
%We used the lumbar vertebraes L2 to L4 to calculate the dose to the red marrow.

\subsection*{Methods}

For the dosimetry, we used an image based approach.
We started with scatter and attenuation corrected SPECT/CT-images that we calibrated to use for activity quantification.
For the red marrow we meassured the activity concentration in lumbar vertebrae L2, L3 and L4, as these are marrow rich bone sites with a simple geometry.
The volumes of the bone sites were corrected for trabecular bone using tabulated values.
Tumour volumes of interest were drawn in a slice by slice manner together with a nuclear medicine specialist.
Activity at different time points were fitted to time activity curves and integrated.
After calculating the time integrated activity coeffients we used OLINDA/EXM to calculate the dose to tumours and red marrow, using the OLINDA sphere model for the tumour and a mean S-value for marrow to marrow and remainder of body to marrow contribution.
So, in this presentation we present the tumour doses, the red marrow doses and the ratio of the patient mean tumour dose and red marrow dose.

\subsection*{Results}

%We combined the two arms that did not get the lilotomab pre-dosing into one group, and compared that group to arm 1 and 4.
%We found that the red marrow dose in arm 4 and 1 were significantly lower than for the non-pre-dosed group.
%The tumour-doses did vary both inter and intra-patiently, and there was not found a significant differences in tumour absorbed doses between the arms.
%To give a meaningful comparison, the mean tumour dose of each patient were diveded by the red marrow absorbed dose, giving a therapeutic ratio.
%This ratio of tumour to red marrow absorbed dose were compared, and the result is shown in the left boxplot.
%We found a significantly higher tumour to red marrow dose in both arm 4 and arm 1 when compared to the lilotomab minus-group.


For the dosimetric analyses, arms 2 and 3 were combined in a single group since neither arms had received pre-dosing with lilotomab.
On the left we see the red marrow absorbed dose for the three patient groups, lilotomab-minus and arm 1 and 4.
We found a significantly lower red marrow absorbed dose for patients that had received lilotomab as pre-dosing. However, the difference between arm 1 and 4, the groups with different amount of lilotomab, was not significant.
On the right we see the tumour doses grouped likewise.
The tumour absorbed doses varied substantially both inter and intra-patiently, and the mean dose was not significantly different between the groups 
Here we see the tumour doses divided by the red marrow absorbed dose.
The ratio of tumour to red marrow absorbed dose was significantly higher in both arm 1 and arm 4 compared to the group that did not receive unlabelled lilotomab.
The mean ratio doubled for arm 1 compared to lilotomab minus, and doubled again for arm 4.


\subsection*{Summary}

So, to summarize:
For all patient arms, red marrow was found to be the primary dose-limiting organ for 177Lu-lilotomab satetraxetan therapy, and pre-dosing with lilotomab had a mitigating effect on red marrow absorbed dose. 

Both pre-dosage levels investigated significantly increased the tumour to red marrow absorbed dose ratio after treatment with 177Lu-lilotomab satetraxetan in patients with non-Hodgkin`s lymphoma. 
Pre-dosing should be considered mandatory in this treatment regimen.


\end{document}
